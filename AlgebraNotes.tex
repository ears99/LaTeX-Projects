\documentclass{article}

\usepackage{amsmath}
\usepackage{tikz}
\usepackage{pgfplots}
\usepackage{graphicx}

\title{Algebra Notes}
\author{Jacob Seadorf}
\graphicspath{{./images/}}

\begin{document}
	\maketitle
\section{Polynomials \& FOIL}
\textbf{FOIL}: An acronym for steps on how to solve a polynomial: First, Outer, Inner, Last
\subsection{Examples}
\subsubsection{Example 1}
$$
(x+2)(3x+5)
$$
Multiply the first terms (x and 3x), the outer terms (x and 5), the inner terms (2 and 3x), and the last terms (2 and 5).
$$
3x^2 + 5x + 6x + 10
$$
Combine like terms
$$
3x^2 + 11x + 10
$$

\subsubsection{Example 2}
$$(2x - 7)(3x + 1)$$
$$6x^2 + 2x - 21x - 7$$
$$6x^2 - 19x - 7$$

\section{Exponent Rules}
\textbf{Note}: All denominators $\ne 0$
\subsection{Product} 
	$$
	x^r \times x^s = x^{r+s}
	$$
\subsection{Quotient}
	$$
	\frac{x^r}{x^s} = x^{r-s}
	$$
\subsection{Power of a Power}
	$$
	(x^r)^s = x^{rs}
	$$
\subsection{Power of a Product}
	$$
	(x \times y)^r = x^r \times y^r
	$$
\subsection{Power of a Quotient}
	$$
	\left(\frac{x}{y}\right)^r = \frac{x^r}{y^r}	
	$$
\subsection{Negative Exponent}
	$$
	x^{-r} = \frac{1}{x^r}
	$$
\subsection{Zero Exponent}
	if $x \ne 0$

	$$x^0 = 1$$

\subsection{Negative Exponent with Quotient}
	$$
	\left( \frac{x}{y}\right)^{-r} = \left(\frac{x}{y}\right)^r
	$$

\subsection{Fractional}
	$$
	a^\frac{1}{n} = \sqrt[n]{a}	
	$$

\section{Factoring \& GCF}
\noindent \textbf{Greatest Common Factor}: The largest number that's common to two or more factors.
\subsection{Examples}
\subsubsection{Example 1}
$$4x^2 - 12x$$
The GCF is 4x
$$4x(x-3)$$

\subsubsection{Example 2}
$$x^2 + 7x + 12$$
Take out the $x$ from the expression.
$$(x)(x)$$
Find two numbers that \textit{multiply} to get 12, and \textit{add} to get 7.
$$4, 3$$
Add them to the factored form of the expression.
$$(x + 4)(x + 3)$$
Use FOIL to check your answer.

\subsubsection{Example 3}
$$x^2 + 2x - 15$$
$$(x+5)(x-3)$$

\section{Finding Slope \& Other Line Equations}
\subsection{Finding Slope from a Graph}

\begin{center} \begin{tikzpicture}
	\begin{axis}[xmin=-10,xmax=10,ymin=-10,ymax=10,axis x line=middle, axis y line=middle];
	\addplot[domain=-10:10]{x+2};	
	\end{axis}
\end{tikzpicture} \end{center}

$$slope = \frac{rise}{run} = \frac{y_2 - y_1}{x_2 - x_1}$$

Choose any two points on the line, $(x_1, y_1)$, $(x_2, y_2)$, and plug them into the equation shown above.

For example, let's choose $(0,2)$ and $(2,4)$.
$$ m = \frac{4 - 2}{2 - 0}$$
Doing the math, you'll see that the slope is $\frac{2}{2}$ or, simplifying the fraction, 1.

\subsection{Finding slope from slope-intercept form}
	$$y = \underline{m}x+b$$
\subsection{Examples}
\subsubsection{Example 1}
$$y = 9x - 4$$
9 is the slope, -4 is the y-intercept

\section{Squaring Binomials}

$$(a \pm b)^2 \Rightarrow a^2 \pm 2ab + b^2$$  
OR
$$(a \pm b)^2 \Rightarrow (a \pm b)(a \pm b)$$

\subsection{Examples}
\subsubsection{Example 1}

$$(2x + 5)^2$$
$$(2x)^2 + 2(2x)(5) + (5)^2$$
$$4x^2 + 20x + 25$$

\subsubsection{Example 2}
$$ (3y^2 - 7w)^2 $$
$$ (3y^2)^2 - 2(3y^2)(7w) + (7w)^2$$
$$ 9y^4 - 42wy^2 + 49w^2 $$ 

\section{Sets, Set Notation, \& Interval Notation}
\noindent \textbf{Set}: A collection of objects (called ``elements'', ``members'', or ``points'').

The Empty Set (the set with nothing in it) is denoted with $\emptyset$. 
\subsection{Set Builder Notation}
The set of all elements $x$ satisfying property $P$ is denoted as $\{x | P(x)\}$.

$|$ means ``such that'' \\ 
$\in$ means ``is a member of''\\
$\notin$ means ``not a member of''\\
There are no repeated elements in a set. A collection of numbers 1, 1, 1, 4, 5 is denoted as $\{1,4,5\}$.
\subsection{Interval Notation}
A square bracket includes a number, a parenthesis excludes a number.

All real numbers $\geq$ 2 and $<$ 4 is denoted as $[2, 4)$.
All real numbers $\geq$ 2 is denoted as $[2,\infty)$.

\section{0, 1, or Infinite Solutions}

\begin{itemize}
\item An equation with no solution is false, and doesn't involve the variable.

\item An equation with only one solution is true, and the variable equals the answer.

\item An equation with infinite solutions is true, but doesn't involve the variable.
\end{itemize}

\subsection{Examples}
\subsubsection{Example 1}
$$ 2x - 3 = 2x - 1$$
$$2x - 2x = -1 + 3$$
$$0 = 2$$
The answer doesn't make sense; no solution.
\subsubsection{Example 2}
$$2y - 4 = y - 2$$
$$2y - y = -2 + 4$$
$$y = 2$$
This equation only has one solution; to check the answer, plug in the value for $y$.
\subsubsection{Example 3}
$$3w - 1 = 3w - 1$$
$$3w - 3w = 1 - 1$$
$$ 0 = 0$$
Infinite solutions; for any value we plug into $w$, the answer will \textit{always} be true

\section{Solving Systems of Linear Equations}
There are several methods of solving systems of linear equations: 
\begin{itemize}
	\item Graphing
	\item Substitution
	\item Elimination
\end{itemize}
\subsection{Solve by Graphing}
Consider the following system: 

$$y = x + 3$$
$$y = -x - 3$$
To solve by graphing, all we would have to do is graph each line, and the solution is where the lines intersect (in this case, $(-3,0)$ -- check your work by plugging it back into the equation.
\subsection{Solve by Substitution}
To solve by substitution, you have to simplify one equation, and incorporate it into the other, which allows you to eliminate one of the variables.
$$3x + y = 6$$
$$x = 18 - 3y$$
In this case, $x$ is already isolated in the second equation, so we can substitute $x$ in the first equation $(18-3y)$.
\subsection{Solve by Elimination}
To solve by elimination, write the equations next to each other so that you can compare the coefficients with each variable.
$$x + y = 180$$
$$3x + 2y = 414$$
Multiply the first equation by $-3$...
$$-3(x + y = 180) \Rightarrow -3x + -y = -540$$
Why did we multiply by -3? Add the first equation to the second, and you get: 
$$0 + -1y = -126$$
Then it's as easy as solving for $y$.
$$y = 126$$
Now plug in $y$ to find $x$...
$$x + y = 180$$
$$x + 126 = 180$$
$$x = 54$$

\section{Factorization Formulas}
\begin{itemize}
	\item \textbf{Difference of Squares}: $a^2 - b^2 = (a + b)(a - b)$
	\item \textbf{Perfect Square}: $a^2 \pm 2ab + b^2 = (a \pm b)^2$
	\item \textbf{Difference/Sum of Cubes}: $a^3 \pm b^3 = (a \pm b)(a^2 \pm ab + b^2)$
	\item \textbf{Perfect Cubes}: $a^3 \pm 3a^2b + 3ab^2 \pm b^3 = (a \pm b)^3$
\end{itemize}

\section{Parabolas}
For the quadratic equation $ax^2 + bx + c = 0$:
If $a > 0$, the parabola opens \textit{upward}.
\noindent If $a < 0$ the parabola opens \textit{downward}.
\noindent The vertex of a parabola is the highest or lowest point.


\section{Complex Numbers}
Normally, a square root of a negative number (for example, $\sqrt{-16}$) is undefined, but there's a way around this: a special type of number was invented by mathematicians to solve this problem, represented by the letter $i$.

$$i = \sqrt{-1}$$

\subsection{Examples}
\subsubsection{Example 1}
$$(-4 + 5i) - (-5 + 2i)$$
$$-4 + 5i + 5 - 2i$$
$$1 + 3i$$

\section{Logarithms}
\textbf{Euler's Number}: A constant represented by the letter $e$, equal to approximately 2.71... \\
\textbf{Base}: The number written under the $\log$. If there's no number written under $\log$, the base is 10. \\

\noindent We all know that operations in math have inverses; addition can be undone by subtraction, multiplication can be undone by division, but what about exponents? That's where logarithms come in. The basic rule for logarithms is: 

\begin{center}
	$$\log_a c = b$$ 
	if and only if 
	$$a^b = c$$
\end{center}

There's several forms of logarithm: 
\begin{itemize}
	\item Base-10 Logarithm, usually just written as $\log$.
	\item Natural Logarithm, with base $e$, written as $\ln$.
	\item Base-a Logarithm, with base $a$, written as $\log_a$.
\end{itemize}

A base-a logarithm of $x$ is the exponent to which $a$ must be raised to get $x$
$$\log_a x = s$$

\subsection{Examples}
\subsubsection{Example 1}
$$2^3 = 8 \Leftrightarrow \log_2 8 = 3$$

\subsubsection{Example 2}
$$5^2 = 25 \Leftrightarrow \log_5 25 = 2$$

\subsubsection{Example 3}
How many 5s need to be multiplied together to get 625?
$$\log_5(625)$$
$$5^4 = 625$$
$$\log_5 625 = 4$$

\section{Properties of Logarithms}
\begin{itemize}
	\item Logarithm of a Product: $\log_a(xy) = \log_a x + \log_a y$
	\item Logarithm of a Power: $\log_a x^n = n \times \log_a x$
	\item Change of Base Formula: $\log_a (x) = \frac{\log_c a}{\log_c x}$
	\item Natural Logarithms: $\ln e = 1$
\end{itemize}

\section{Matrices \& Matrix Operations}
\textbf{Matrix}: A rectangular array or table of numbers, symbols, or expressions.

\begin{center}
\begin{bmatrix}
	1 & 9 & -13 \\ 
	20 & 5 & -6
\end{bmatrix}
\end{center}

\subsection{Matrix Operations}
\subsubsection{Inverse}
	The inverse of a matrix is just the opposite of a matrix. 
	$$
	-
	\begin{bmatrix}
		a & -b \\
		c & d
	\end{bmatrix}
	=
	\begin{bmatrix}
		-a & b \\
		-c & -d
	\end{bmatrix}
	$$
\subsubsection{Addition/Subtraction}
	For addition and subtraction, add/subtract in the matching positions.
	$$
	\begin{bmatrix}
		\underline{a} & b \\ 
		c & d 
	\end{bmatrix}
	\pm
	\begin{bmatrix}
		\underline{e} & f \\ 
		g & h
	\end{bmatrix}
	$$
\subsubsection{Scaler Multiplication}
	For scaler multiplication, simply multiply each element by the scaler.
	$$
	a
	\times
	\begin{bmatrix}
		b & c \\
		d & e
	\end{bmatrix}
	= 
	\begin{bmatrix}
		ab & ac \\
		ad & ae
	\end{bmatrix}
	$$

\section{Matrix Multiplication}
You multiply matrices by taking the dot product of each matrix. 
$$
\begin{bmatrix}
	a & b & c \\
	d & e & f
\end{bmatrix}
\times
\begin{bmatrix}
	g & j \\
	h & k \\
	i & l
\end{bmatrix}
= 
\begin{bmatrix}
ag & + & bh & + & ci \\
aj & + & bk & + & cl \\
da & + & eh & + & fi \\
dj & + & ek & + & fl
\end{bmatrix}
$$
It should be noted that matrices A and B are \textbf{not} commutative; 
$$AB \ne BA$$

\section{Sigma Notation}
The Greek letter $\Sigma$ has a special meaning in math equations: summation. 
Suppose you wanted to add up a sequence of numbers, one after another. You \textit{could} write it out:
$$1 + 2 + 3 + 4...$$

But that would take a lot of time if you had a lot of numbers that needed adding! So mathematicians use the letter $\Sigma$ to represent this. 

You sum whatever is after the $\Sigma$:
$$\sum n$$

But what \textit{values} of $n$?
$$\sum_{i=1}^{4} n$$
You start at the number at the bottom (1, in this case) and add up to the number at the top (4, in this case).

But $\Sigma$ is a \textit{lot} more powerful than that!

We can square $n$ each time and add the result: 
$$\sum_{i=1}^{4} n^2 = 1^2 + 2^2 + 3^2 + 4^2 = 30$$

We can add up the first four terms of the expression $2n+1$:
$$\sum_{i=1}^{4} (2n+1) = 3 + 5 + 7 + 9 = 24$$

\section{Function Composition}
Suppose that you're a farmer, and you plant seeds that turn into corn. You write a function to determine how much corn, $C$, in kilograms, that you'll produce after planting in a given number of acres of land, $a$:

$$C(a) = 7500a - 1500$$

For example, if you plant 2 acres of corn, you'll get 13,500 kilograms: 

$$C(2) = 7500(2) - 1500 = 13,500$$ 

What you \emph{really} want to know, however, is how much money you'll get after selling your corn, so you write another function, $M$: 

$$M(c) = 0.9c - 50$$

If you produce 13,500 kilograms of corn, you can expect to make \$12,100: 

$$M(13,500) = 0.9(13,500) - 50 = \$12,100$$

This is all well and good, but we're efficient farming mathematicians -- we want to create a function that directly converts planted acres to our earnings. To find this function, let's think about the most general question: how much money do we expect to make if we plant on $a$ acres of land? Well, if we plant corn on $a$ acres, we expect to produce $C(a)$ kilograms of corn, and if we produce $C(a)$ kilograms of corn, we expect to make $M(C(a))$ dollars.

So to find a general rule that converts acres of land into the expected earnings, we can find the expression
$$M(C(a))$$

But how do we do this? Notice that in the expression $M(C(a))$, the input of the function $M$ is $C(a)$. To find this expression, we can substitute $C(a)$ in for $c$ in function $M$: 

$$M(c) = 0.9c - 50$$
$$M(C(a)) = 0.9(C(a)) - 50$$
$$0.9(7500a - 1500) - 50$$
$$6750a - 1350 - 50$$
$$6750a - 1400$$
So the function $M(C(a)) = 6750a - 1400$ converts acres planted directly into expected earnings. We can use this function to check how much money you'll earn when planting on 2 acres: 

$$M(C(2)) = 6750(2) - 1400 = \$12,100$$
We can expect to make \$12,100 from planting on 2 acres of land, which matches up with our previous work. 

We just found what's called a \textbf{composite function}. Instead of substituting acres planted into the corn function, and then substituting the amount of corn produced into the money function, we found a function that takes the acres planted directly to the expected earnings. 

We did this by substituting $C(a)$ into function $M$, or by finding $M(C(a))$. Let's call this new function $M \circ C$, which is read as "M composed with C''. 
We now know that $(M \circ C)(a) = M(C(a))$, which is the formal definition of function composition. 
\subsection{Examples}
Ben is a potato farmer. The function $P(a) = 25,000a - 1,000$ gives the amount of potatoes, $P$, in kilograms, that he expects to produce from planting potatoes on $a$ acres of land. The function $M(p) = 0.2p - 200$ gives the amount of money, $M$, in dollars, that Ben expects to make if he produces $p$ kilograms of potatoes. How much money can Ben expect to make if he sells all the potatoes produced on 3 acres? 

$$M(P(a)) = 0.2(P(a)) - 200$$
$$M(P(a)) = 0.2(25,000a - 1,000) - 200$$
$$M(P(a)) = 5000a - 200 - 200$$
$$M(P(a)) = 5000a - 400$$
Having created our composite function, we can now answer the question: 
$$M(P(3)) = 5,000(3) - 400 = 14,600$$
Ben can expect to make \$14,600 if he sells all the potatoes produced on 3 acres of land.
\end{document}
